\documentclass[twoside]{article}\usepackage[]{graphicx}\usepackage[]{color}
% maxwidth is the original width if it is less than linewidth
% otherwise use linewidth (to make sure the graphics do not exceed the margin)
\makeatletter
\def\maxwidth{ %
  \ifdim\Gin@nat@width>\linewidth
    \linewidth
  \else
    \Gin@nat@width
  \fi
}
\makeatother

\definecolor{fgcolor}{rgb}{0.345, 0.345, 0.345}
\newcommand{\hlnum}[1]{\textcolor[rgb]{0.686,0.059,0.569}{#1}}%
\newcommand{\hlstr}[1]{\textcolor[rgb]{0.192,0.494,0.8}{#1}}%
\newcommand{\hlcom}[1]{\textcolor[rgb]{0.678,0.584,0.686}{\textit{#1}}}%
\newcommand{\hlopt}[1]{\textcolor[rgb]{0,0,0}{#1}}%
\newcommand{\hlstd}[1]{\textcolor[rgb]{0.345,0.345,0.345}{#1}}%
\newcommand{\hlkwa}[1]{\textcolor[rgb]{0.161,0.373,0.58}{\textbf{#1}}}%
\newcommand{\hlkwb}[1]{\textcolor[rgb]{0.69,0.353,0.396}{#1}}%
\newcommand{\hlkwc}[1]{\textcolor[rgb]{0.333,0.667,0.333}{#1}}%
\newcommand{\hlkwd}[1]{\textcolor[rgb]{0.737,0.353,0.396}{\textbf{#1}}}%
\let\hlipl\hlkwb

\usepackage{framed}
\makeatletter
\newenvironment{kframe}{%
 \def\at@end@of@kframe{}%
 \ifinner\ifhmode%
  \def\at@end@of@kframe{\end{minipage}}%
  \begin{minipage}{\columnwidth}%
 \fi\fi%
 \def\FrameCommand##1{\hskip\@totalleftmargin \hskip-\fboxsep
 \colorbox{shadecolor}{##1}\hskip-\fboxsep
     % There is no \\@totalrightmargin, so:
     \hskip-\linewidth \hskip-\@totalleftmargin \hskip\columnwidth}%
 \MakeFramed {\advance\hsize-\width
   \@totalleftmargin\z@ \linewidth\hsize
   \@setminipage}}%
 {\par\unskip\endMakeFramed%
 \at@end@of@kframe}
\makeatother

\definecolor{shadecolor}{rgb}{.97, .97, .97}
\definecolor{messagecolor}{rgb}{0, 0, 0}
\definecolor{warningcolor}{rgb}{1, 0, 1}
\definecolor{errorcolor}{rgb}{1, 0, 0}
\newenvironment{knitrout}{}{} % an empty environment to be redefined in TeX

\usepackage{alltt}
\usepackage[utf8]{inputenc}
\usepackage[czech]{babel}
\usepackage{fancyhdr}
\usepackage{amsmath}
\usepackage{amsfonts}
\usepackage[paper=a4paper, nomarginpar, foot=1.5cm, top=2.5cm, bottom=2.5cm, left=2.5cm, right=2.5cm]{geometry}
\usepackage{siunitx}

\pagestyle{fancy}
\fancyhead{} % clear all header fields
\fancyhead[RO,LE]{Marek Földi}
\fancyhead[RE,LO]{Úkol 6}
\fancyfoot{} % clear all footer fields
\fancyfoot[LE,RO]{\thepage}

\addto\captioneurosyfilisczech{\renewcommand{\figurename}{Graf. č.}}
\IfFileExists{upquote.sty}{\usepackage{upquote}}{}
\begin{document}





\subsection*{Příklad 1:}
\begin{knitrout}
\definecolor{shadecolor}{rgb}{0.969, 0.969, 0.969}\color{fgcolor}\begin{kframe}
\begin{alltt}
\hlstd{poz} \hlkwb{<-} \hlkwd{c}\hlstd{(}\hlnum{24}\hlstd{,} \hlnum{9}\hlstd{,} \hlnum{13}\hlstd{,} \hlnum{22}\hlstd{,} \hlnum{9}\hlstd{)}
\hlstd{pravd} \hlkwb{<-} \hlkwd{c}\hlstd{(}\hlnum{0.21}\hlstd{,} \hlnum{0.15}\hlstd{,} \hlnum{0.26}\hlstd{,} \hlnum{0.16}\hlstd{,} \hlnum{0.22}\hlstd{)}
\hlkwd{chisq.test}\hlstd{(poz,} \hlkwc{p} \hlstd{= pravd)}
\end{alltt}
\begin{verbatim}
## 
## 	Chi-squared test for given probabilities
## 
## data:  poz
## X-squared = 18.143, df = 4, p-value = 0.001157
\end{verbatim}
\begin{alltt}
\hlkwd{chisq.test}\hlstd{(poz,} \hlkwc{p} \hlstd{= pravd)}\hlopt{$}\hlstd{expected}
\end{alltt}
\begin{verbatim}
## [1] 16.17 11.55 20.02 12.32 16.94
\end{verbatim}
\end{kframe}
\end{knitrout}
Rozdělení pacientů s~migrénou je jiné než rozdělení obyvatel do okresů. V~okresu A~i D je výrazně více pacientů s~migrénou než bychom očekávali a u~ostatních okresů je méně pacientů než bychom očekávali. Je možné že tento jev bude souviset s~vyšším znečištěním těchto okresů, ale na to je potřeba další šetření.

\subsection*{Příklad 2:}
\begin{knitrout}
\definecolor{shadecolor}{rgb}{0.969, 0.969, 0.969}\color{fgcolor}\begin{kframe}
\begin{alltt}
\hlstd{tab2} \hlkwb{<-} \hlkwd{matrix}\hlstd{(}\hlkwd{c}\hlstd{(}\hlnum{25}\hlstd{,} \hlnum{19}\hlstd{,} \hlnum{10}\hlstd{,} \hlnum{18}\hlstd{),}\hlnum{2}\hlstd{,}\hlnum{2}\hlstd{,} \hlkwc{byrow} \hlstd{= T)}
\hlkwd{mcnemar.test}\hlstd{(tab2 ,}\hlkwc{correct} \hlstd{= F)}
\end{alltt}
\begin{verbatim}
## 
## 	McNemar's Chi-squared test
## 
## data:  tab2
## McNemar's chi-squared = 2.7931, df = 1, p-value = 0.09467
\end{verbatim}
\end{kframe}
\end{knitrout}
Lék se v testovaném vzorku neprokázal jako výrazně účinný na to abychom nevyloučili, že spíše působil náhodou.

\subsection*{Příklad 3:}
\begin{knitrout}
\definecolor{shadecolor}{rgb}{0.969, 0.969, 0.969}\color{fgcolor}\begin{kframe}
\begin{alltt}
\hlstd{tab3} \hlkwb{<-} \hlkwd{matrix}\hlstd{(}\hlkwd{c}\hlstd{(}\hlnum{31}\hlstd{,} \hlnum{23}\hlstd{,} \hlnum{38}\hlstd{,} \hlnum{11}\hlstd{,} \hlnum{17}\hlstd{,} \hlnum{8}\hlstd{,} \hlnum{8}\hlstd{,} \hlnum{10}\hlstd{,} \hlnum{4}\hlstd{),} \hlnum{3}\hlstd{,} \hlnum{3}\hlstd{,} \hlkwc{byrow} \hlstd{= T)}
\hlkwd{dimnames}\hlstd{(tab3)} \hlkwb{<-} \hlkwd{list}\hlstd{(}\hlkwc{rows} \hlstd{=} \hlkwd{c}\hlstd{(}\hlstr{"1"}\hlstd{,} \hlstr{"2"}\hlstd{,} \hlstr{"3"}\hlstd{),} \hlkwc{columns} \hlstd{=} \hlkwd{c}\hlstd{(}\hlstr{"1"}\hlstd{,} \hlstr{"2"}\hlstd{,} \hlstr{"3"}\hlstd{))}
\hlstd{tab3}
\end{alltt}
\begin{verbatim}
##     columns
## rows  1  2  3
##    1 31 23 38
##    2 11 17  8
##    3  8 10  4
\end{verbatim}
\begin{alltt}
\hlstd{test3} \hlkwb{<-} \hlkwd{chisq.test}\hlstd{(tab3,} \hlkwc{correct} \hlstd{= F)}
\hlstd{test3}
\end{alltt}
\begin{verbatim}
## 
## 	Pearson's Chi-squared test
## 
## data:  tab3
## X-squared = 9.7194, df = 4, p-value = 0.04543
\end{verbatim}
\begin{alltt}
\hlstd{test3}\hlopt{$}\hlstd{expected}
\end{alltt}
\begin{verbatim}
##     columns
## rows         1         2         3
##    1 30.666667 30.666667 30.666667
##    2 12.000000 12.000000 12.000000
##    3  7.333333  7.333333  7.333333
\end{verbatim}
\begin{alltt}
\hlkwd{remove}\hlstd{(test3)}
\hlkwd{fisher.test}\hlstd{(tab3)}
\end{alltt}
\begin{verbatim}
## 
## 	Fisher's Exact Test for Count Data
## 
## data:  tab3
## p-value = 0.04524
## alternative hypothesis: two.sided
\end{verbatim}
\begin{alltt}
\hlkwd{remove}\hlstd{(tab3)}
\end{alltt}
\end{kframe}
\end{knitrout}
Kofein může souviset s~výsledkem těhotenství, to nám potvrzuje $\text{chi}^2$ test i Fisherův test, ve kterých vyšla p hodnota pod 0,05. Z~porovnání s~předpokládanými výsledky, můžeme usuzovat, že kofein může souviset s~předčasnými porody.
\end{document}
